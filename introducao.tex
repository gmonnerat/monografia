\chapter{Introdução}
Este trabalho é fruto de uma iniciativa entre a Nexedi SA(empresa situada na França) e o NSI(Núcleo de pesquisa de informática no IFF-Campos) de desenvolverem uma ferramenta opensource que subsitui-se uma já existente, chamada OpenOffice.org Daemon (oood), desenvolvida originalmente pela Nexedi SA. A primeira versão do OpenOffice.org Daemon, ou oood, foi desenvolvida em 2006 pela Nexedi com o objetivo inicial de converter documentos em grande escala e ser integrada ao ERP5 DMS. O ERP5 é um framework web de código aberto, também desenvolvido pela Nexedi, que é utilizado para criação de sistemas ERP, CRM, MRP, SCM, PDM para empresas, indústrias e orgãos governamentais. O Document Manager System(DMS) é um sistema utilizado para armazenar, categorizar e buscar documentos eletrônicos, sendo muitas vezes visto como um componente de Enterprise Content Management (ECM). Com isso, o módulo ERP5 DMS através da ferramenta de conversão de documentos pode ter qualquer informação do documento, tornando fácil a busca por metadados, conversão de documentos em formato original para qualquer outro formato suportado pelo Openoffice.org além do formato pdf.

Com o uso massivo de conversões foi identificado erros gerados por bugs ou tratamentos inadequados que ocasionam problemas como: requisições perdidas, deadlock do OpenOffice.org e processos, interrupção do serviço de rede, entre outros. Logo, surgiu a necessidade de corrigir o serviço para torná-lo mais estável e com bom desempenho. A análise preliminar do oood determinou que seria inviável a manutenção do mesmo, pois havia novos requisitos não implementados e necessidade da correção dos problemas identificados.

Com a necessidade de novos requisitos, recursos mais atuais e estáveis, surge a motivação para o desenvolvimento de uma nova ferramenta que tivesse como funcionalidades básicas converter documentos para a base ODF (Open Document Format), exportar documentos para outras extensões, adicionar e extrair metadados de um documento nos formatos suportados pelo OpenOffice.org. Um dos fatores que também influenciaram na iniciativa de uma nova ferramenta foi a busca por uma estrutura mais flexivel, de fácil manutenção e extensão.

Portanto este trabalho tem como objetivo demostrar o desenvolvimento de uma ferramenta que utiliza-se de recursos computacionais para manipulação de documentos, capaz de extrair e adicionar informações e converter para qualquer outro formato suportado pelo Openoffice.org.

No capítulo II deste trabalho apresentamos uma revisão bibliográfica para fundamentar o estudo realizado. O capítulo III demonstra as tecnologias empregada para o desenvolvimento da