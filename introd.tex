\chapter{Introdução}
A Internet, uma das grandes revoluções tecnológicas da comunicação dos últimos dez anos, deve essa popularidade ao avanço na tecnologia Web, que popularizou uma interface gráfica intuitiva baseada em \textit{hiperlinks}, possibilitando a fácil utilização dos serviços disponibilizados via internet. O crescente volume de computadores conectados a rede altera profundamente o modo de desenvolvimento dos aplicativos, obrigando corporações a introduzirem novas tecnologias que suportem a grande demanda. Logo, com o crescimento tanto de usuários quanto de aplicações via rede de computadores, novas dificuldades surgiram com a extensão das aplicações, como maior complexidade de manutenção, a dificuldade de inserir novas funcionalidades e a curva de aprendizado acentuada aos desenvolvedores. Visando minimizar estas dificuldades, surge a necessidade de serviços descentralizados, mais simples e com uma confiança maior para desenvolvimento de softwares em larga escala.

Com a descentralização das informações, o desenvolvimento de sistemas simples e escaláveis fez com que surgissem cada vez mais serviços Web que tivessem somente uma responsabilidade e pudessem se comunicar com outros aplicativos via internet ou intranet. Em resumo, serviços simples, de fácil manutenção e extensão, proporcionam o aumento da perfomace e escalabilidade, eliminando o uso de um serviço único com todo volume de informações para serem processadas constantemente.

A iniciativa entre a empresa francesa Nexedi SA(empresa situada em Lille, França) e o NSI(Núcleo de pesquisa em Sistema de Informação) do Instituto Federal Fluminense-Campos de desenvolverem uma ferramenta Web de código aberto que subsitui-se uma já existente, chamada OpenOffice.org Daemon (oood), desenvolvida originalmente pela Nexedi SA, é um exemplo da necessidade do uso de novas tecnologias para suportar novas demandas e requisitos.

A primeira versão do oood foi desenvolvida em 2006 pela Nexedi com o objetivo inicial de converter documentos do tipo Office. A partir do uso prolongado em ambientes de produção foram identificados erros ou tratamentos inadequados de exceções que ocasionaram problemas como: perda de requisições, \textit{deadlock} no OpenOffice.org e em processos, \textit{memory leak}, entre outros. Logo, surgiu a necessidade de corrigir o serviço para torná-lo mais estável, com um bom desempenho e escalável em forma de cluster. A análise preliminar do oood determinou que seria inviável a manutenção do mesmo, pois havia demanda por novos requisitos e a necessidade de corrigir os problemas mencionados acima.

Com a necessidade da correção dos problemas identificados e o uso de resursos mais atuais e estáveis, surge a motivação para o desenvolvimento de uma nova ferramenta que tivesse como funcionalidades básicas converter documentos para a base ODF (\textit{Open Document Format}), exportar documentos para outras extensões, adicionar e extrair metadados de um documento nos formatos suportados pelo OpenOffice.org. Outros fatores que também influenciaram na iniciativa de uma nova ferramenta foi a busca por uma estrutura mais flexível, de fácil manutenção, extensível e escalável.

Portanto, este trabalho tem como objetivo desenvolver uma ferramenta Web que utilize de recursos computacionais para manipulação de documentos, de forma que seja possível extrair e adicionar informações e fazer conversões destes documentos para qualquer formato suportado pelo Openoffice.org. Além disso, apresentar soluções para os problemas identificados, tais como \textit{deadlock} no OpenOffice.org, perda de requisições e falta de escalabilidade.

No segundo capítulo são apresentados conceitos básicos para melhor entendimento das tecnologias e padrões utilizados durante o trabalho. No terceiro capítulo, é explicado separadamente cada componente desenvolvido com o intuito de apresentar no quarto capítulo estudos de casos descrevendo em detalhes como a junção destes componentes formam um serviço Web e o uso deste em um ambiente similiar ao de produção.

Por fim, o quinto capítulo apresenta os resultados obtidos com o estudo de casos e a conclusão obtida a partir do desenvolvimento desta aplicação.