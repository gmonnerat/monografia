\chapter{Conclusão}
\label{conclusao}
\section{Objetivos alcançados}
Neste trabalho foi apresentado o \textit{Web Service} OOOD 2.0, uma ferramenta \textit{Open Source} para conversão de documentos em larga escala. Sua implementação de forma genérica, fez com que cada parte da aplicação possa ser utilizada separamente como uma biblioteca comum, tornando possível adicionar uma nova implementação caso almeje-se extender para atender requisitos específicos.

Foi visto também que, com o Python e seus recursos, é possível desenvolver ferramentas que suportam grandes demandas sem que a legibilidade e flexibilidade do código sejam esquecidas.

\section{Trabalhos futuros}

Implementar manipuladores de imagens e vídeos para que seja possível converter e extrair metadados destes tipos de arquivos. Além disso, diminuir os dados que são tranferidos entre os componentes da aplicação e os \textit{scripts} \textit{UnoConverter} e \textit{UnoMimeMapper} pois, diminuindo o tamanho dos dados, há um redução direta no tempo de resposta para o cliente.

É necessário também melhorar o ambiente de testes para que não seja necessário iniciar e parar o OpenOffice.org a cada teste, ou seja, para que os testes executem iniciando e parando o OpenOffice.org somente um vez, diminuindo o tempo gasto para executá-los.