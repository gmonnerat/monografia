\section{OpenOffice.org}
\label{openoffice}
OpenOffice.org é uma suíte de aplicativos multiplataforma, sendo distribuída para sistemas operacionais como Linux, Microsoft Windows, Solaris, Mac OS X. A suíte usa os formatos ODF e também é compatível com formatos da Microsoff Office. No caso de documentos que não estão no formato ODF, estes documentos são convertidos para uma extensão de mesmo tipo que seja ODF. O OpenOffice.org divide as extensões dos documentos em tipos, ou seja, cada documento tem o seu tipo e documentos só podem ser convertido para outra extensão que faça parte do seu grupo de tipos. As extensões nas suítes são dividas em 5 tipos onde cade tipo tem a sua extensão padrão, por exemplo um documento de tipo texto é convertido para o formato ODT quando carregado na suíte. Com essa padronização, documentos criados e suportados pela suíte são compátiveis com suítes instaladas em outro sistema operacional.

Inicialmente o OpenOffice.org era chamado de StarOffice. A Sun MicroSystems comprou este na versão 5.1, em 1999, da empresa alemã StarDivison. Em agosto de 1999, a Sun disponibiliza gratuitamente a versão 5.2 do StarOffice e em meados de 2000 anuncia a liberação do código fonte para download sob as licenças LGPL e SISSL com o intuito de criar uma alternativa de baixo custo, código aberto e de alta qualidade. LGPL e SISSL impõe que o código aberto desenvolvido esteja disponivel em forma de bibliocata, mas não exige que o mesmo seja aplicado a outros softwares que empreguem seu código. No mesmo ano da liberação da versão 5.2 do StarOffice, o novo projeto recebe o nome de OpenOffice.org e no Brasil de BrOffice.org.